% Options for packages loaded elsewhere
\PassOptionsToPackage{unicode}{hyperref}
\PassOptionsToPackage{hyphens}{url}
%
\documentclass[
]{article}
\usepackage{amsmath,amssymb}
\usepackage{iftex}
\ifPDFTeX
  \usepackage[T1]{fontenc}
  \usepackage[utf8]{inputenc}
  \usepackage{textcomp} % provide euro and other symbols
\else % if luatex or xetex
  \usepackage{unicode-math} % this also loads fontspec
  \defaultfontfeatures{Scale=MatchLowercase}
  \defaultfontfeatures[\rmfamily]{Ligatures=TeX,Scale=1}
\fi
\usepackage{lmodern}
\ifPDFTeX\else
  % xetex/luatex font selection
\fi
% Use upquote if available, for straight quotes in verbatim environments
\IfFileExists{upquote.sty}{\usepackage{upquote}}{}
\IfFileExists{microtype.sty}{% use microtype if available
  \usepackage[]{microtype}
  \UseMicrotypeSet[protrusion]{basicmath} % disable protrusion for tt fonts
}{}
\makeatletter
\@ifundefined{KOMAClassName}{% if non-KOMA class
  \IfFileExists{parskip.sty}{%
    \usepackage{parskip}
  }{% else
    \setlength{\parindent}{0pt}
    \setlength{\parskip}{6pt plus 2pt minus 1pt}}
}{% if KOMA class
  \KOMAoptions{parskip=half}}
\makeatother
\usepackage{xcolor}
\usepackage[margin=1in]{geometry}
\usepackage{color}
\usepackage{fancyvrb}
\newcommand{\VerbBar}{|}
\newcommand{\VERB}{\Verb[commandchars=\\\{\}]}
\DefineVerbatimEnvironment{Highlighting}{Verbatim}{commandchars=\\\{\}}
% Add ',fontsize=\small' for more characters per line
\usepackage{framed}
\definecolor{shadecolor}{RGB}{248,248,248}
\newenvironment{Shaded}{\begin{snugshade}}{\end{snugshade}}
\newcommand{\AlertTok}[1]{\textcolor[rgb]{0.94,0.16,0.16}{#1}}
\newcommand{\AnnotationTok}[1]{\textcolor[rgb]{0.56,0.35,0.01}{\textbf{\textit{#1}}}}
\newcommand{\AttributeTok}[1]{\textcolor[rgb]{0.13,0.29,0.53}{#1}}
\newcommand{\BaseNTok}[1]{\textcolor[rgb]{0.00,0.00,0.81}{#1}}
\newcommand{\BuiltInTok}[1]{#1}
\newcommand{\CharTok}[1]{\textcolor[rgb]{0.31,0.60,0.02}{#1}}
\newcommand{\CommentTok}[1]{\textcolor[rgb]{0.56,0.35,0.01}{\textit{#1}}}
\newcommand{\CommentVarTok}[1]{\textcolor[rgb]{0.56,0.35,0.01}{\textbf{\textit{#1}}}}
\newcommand{\ConstantTok}[1]{\textcolor[rgb]{0.56,0.35,0.01}{#1}}
\newcommand{\ControlFlowTok}[1]{\textcolor[rgb]{0.13,0.29,0.53}{\textbf{#1}}}
\newcommand{\DataTypeTok}[1]{\textcolor[rgb]{0.13,0.29,0.53}{#1}}
\newcommand{\DecValTok}[1]{\textcolor[rgb]{0.00,0.00,0.81}{#1}}
\newcommand{\DocumentationTok}[1]{\textcolor[rgb]{0.56,0.35,0.01}{\textbf{\textit{#1}}}}
\newcommand{\ErrorTok}[1]{\textcolor[rgb]{0.64,0.00,0.00}{\textbf{#1}}}
\newcommand{\ExtensionTok}[1]{#1}
\newcommand{\FloatTok}[1]{\textcolor[rgb]{0.00,0.00,0.81}{#1}}
\newcommand{\FunctionTok}[1]{\textcolor[rgb]{0.13,0.29,0.53}{\textbf{#1}}}
\newcommand{\ImportTok}[1]{#1}
\newcommand{\InformationTok}[1]{\textcolor[rgb]{0.56,0.35,0.01}{\textbf{\textit{#1}}}}
\newcommand{\KeywordTok}[1]{\textcolor[rgb]{0.13,0.29,0.53}{\textbf{#1}}}
\newcommand{\NormalTok}[1]{#1}
\newcommand{\OperatorTok}[1]{\textcolor[rgb]{0.81,0.36,0.00}{\textbf{#1}}}
\newcommand{\OtherTok}[1]{\textcolor[rgb]{0.56,0.35,0.01}{#1}}
\newcommand{\PreprocessorTok}[1]{\textcolor[rgb]{0.56,0.35,0.01}{\textit{#1}}}
\newcommand{\RegionMarkerTok}[1]{#1}
\newcommand{\SpecialCharTok}[1]{\textcolor[rgb]{0.81,0.36,0.00}{\textbf{#1}}}
\newcommand{\SpecialStringTok}[1]{\textcolor[rgb]{0.31,0.60,0.02}{#1}}
\newcommand{\StringTok}[1]{\textcolor[rgb]{0.31,0.60,0.02}{#1}}
\newcommand{\VariableTok}[1]{\textcolor[rgb]{0.00,0.00,0.00}{#1}}
\newcommand{\VerbatimStringTok}[1]{\textcolor[rgb]{0.31,0.60,0.02}{#1}}
\newcommand{\WarningTok}[1]{\textcolor[rgb]{0.56,0.35,0.01}{\textbf{\textit{#1}}}}
\usepackage{graphicx}
\makeatletter
\def\maxwidth{\ifdim\Gin@nat@width>\linewidth\linewidth\else\Gin@nat@width\fi}
\def\maxheight{\ifdim\Gin@nat@height>\textheight\textheight\else\Gin@nat@height\fi}
\makeatother
% Scale images if necessary, so that they will not overflow the page
% margins by default, and it is still possible to overwrite the defaults
% using explicit options in \includegraphics[width, height, ...]{}
\setkeys{Gin}{width=\maxwidth,height=\maxheight,keepaspectratio}
% Set default figure placement to htbp
\makeatletter
\def\fps@figure{htbp}
\makeatother
\setlength{\emergencystretch}{3em} % prevent overfull lines
\providecommand{\tightlist}{%
  \setlength{\itemsep}{0pt}\setlength{\parskip}{0pt}}
\setcounter{secnumdepth}{-\maxdimen} % remove section numbering
\ifLuaTeX
  \usepackage{selnolig}  % disable illegal ligatures
\fi
\IfFileExists{bookmark.sty}{\usepackage{bookmark}}{\usepackage{hyperref}}
\IfFileExists{xurl.sty}{\usepackage{xurl}}{} % add URL line breaks if available
\urlstyle{same}
\hypersetup{
  pdftitle={Untitled},
  pdfauthor={FC},
  hidelinks,
  pdfcreator={LaTeX via pandoc}}

\title{Untitled}
\author{FC}
\date{2023-11-05}

\begin{document}
\maketitle

\begin{Shaded}
\begin{Highlighting}[]
\CommentTok{\# tidyverse includes dplyr and ggplot2 so I don\textquotesingle{}t need to load them separately}
\FunctionTok{library}\NormalTok{(tidyverse)}
\end{Highlighting}
\end{Shaded}

\begin{verbatim}
## -- Attaching core tidyverse packages ------------------------ tidyverse 2.0.0 --
## v dplyr     1.1.3     v readr     2.1.4
## v forcats   1.0.0     v stringr   1.5.0
## v ggplot2   3.4.4     v tibble    3.2.1
## v lubridate 1.9.3     v tidyr     1.3.0
## v purrr     1.0.2     
## -- Conflicts ------------------------------------------ tidyverse_conflicts() --
## x dplyr::filter() masks stats::filter()
## x dplyr::lag()    masks stats::lag()
## i Use the conflicted package (<http://conflicted.r-lib.org/>) to force all conflicts to become errors
\end{verbatim}

\begin{Shaded}
\begin{Highlighting}[]
\FunctionTok{library}\NormalTok{(here)}
\end{Highlighting}
\end{Shaded}

\begin{verbatim}
## here() starts at C:/Users/Fionnuala/OneDrive - University of Aberdeen/PU5063 Intro to HDS/Assessment
\end{verbatim}

\begin{Shaded}
\begin{Highlighting}[]
\FunctionTok{library}\NormalTok{(tinytex)}
\end{Highlighting}
\end{Shaded}

\hypertarget{question}{%
\section{Question}\label{question}}

What are the regional trends for the percentage of the population
prescribed drugs for anxiety, depression and psychosis in Scotland over
the last ten years? What might these mean for employers' allocation of
support resources? The next sections follow the Health Data Science
Workflow to address these questions.

\hypertarget{data-acquisition}{%
\section{Data Acquisition}\label{data-acquisition}}

The data was downloaded from
\url{https://scotland.shinyapps.io/ScotPHO_profiles_tool/} on 05/11/23
for the item ``population prescribed drugs for
anxiety/depression/psychosis'' for all available years and all health
boards. The downloaded file was called timetrend\_data.csv and for the
purposes of this question, it was renamed adp\_data

\begin{Shaded}
\begin{Highlighting}[]
\CommentTok{\#reading in the data:}
\NormalTok{adp\_data }\OtherTok{\textless{}{-}} \FunctionTok{read\_csv}\NormalTok{(}\FunctionTok{here}\NormalTok{(}\StringTok{"Inputs/timetrend\_data.csv"}\NormalTok{))}
\end{Highlighting}
\end{Shaded}

\begin{verbatim}
## Rows: 180 Columns: 12
## -- Column specification --------------------------------------------------------
## Delimiter: ","
## chr (7): indicator, area_name, area_code, area_type, period, definition, dat...
## dbl (5): year, numerator, measure, lower_confidence_interval, upper_confiden...
## 
## i Use `spec()` to retrieve the full column specification for this data.
## i Specify the column types or set `show_col_types = FALSE` to quiet this message.
\end{verbatim}

\begin{Shaded}
\begin{Highlighting}[]
\FunctionTok{glimpse}\NormalTok{(adp\_data)}
\end{Highlighting}
\end{Shaded}

\begin{verbatim}
## Rows: 180
## Columns: 12
## $ indicator                 <chr> "Population prescribed drugs for anxiety/dep~
## $ area_name                 <chr> "Scotland", "NHS Ayrshire & Arran", "NHS Bor~
## $ area_code                 <chr> "S00000001", "S08000015", "S08000016", "S080~
## $ area_type                 <chr> "Scotland", "Health board", "Health board", ~
## $ year                      <dbl> 2010, 2010, 2010, 2010, 2010, 2010, 2010, 20~
## $ period                    <chr> "2010/11 financial year", "2010/11 financial~
## $ numerator                 <dbl> 787040, 60822, 17226, 22280, 55334, 43976, 7~
## $ measure                   <dbl> 14.96, 16.31, 15.15, 14.75, 15.26, 14.86, 12~
## $ lower_confidence_interval <dbl> 14.93, 16.20, 14.94, 14.57, 15.14, 14.73, 12~
## $ upper_confidence_interval <dbl> 14.99, 16.43, 15.36, 14.92, 15.38, 14.98, 12~
## $ definition                <chr> "Percentage", "Percentage", "Percentage", "P~
## $ data_source               <chr> "Public Health Scotland (Prescribing Informa~
\end{verbatim}

\#Prepare/Clean Data

\begin{Shaded}
\begin{Highlighting}[]
\CommentTok{\# There are no missing values}
\CommentTok{\# This chunk is for selecting and renaming columns, removing the rows for the whole of Scotland and removing the NHS prefix.}
\CommentTok{\# The mutate line was suggested by chatgpt when I gave it the preceding lines in this chunk and asked it what to add to strip out the "NHS " prefix. }
\NormalTok{plot\_data }\OtherTok{\textless{}{-}}\NormalTok{ adp\_data }\SpecialCharTok{\%\textgreater{}\%}
  \FunctionTok{select}\NormalTok{(}\StringTok{\textquotesingle{}area\_name\textquotesingle{}}\NormalTok{,}\StringTok{\textquotesingle{}year\textquotesingle{}}\NormalTok{,}\StringTok{\textquotesingle{}numerator\textquotesingle{}}\NormalTok{) }\SpecialCharTok{\%\textgreater{}\%} 
  \FunctionTok{rename}\NormalTok{(}\AttributeTok{number =} \StringTok{\textquotesingle{}numerator\textquotesingle{}}\NormalTok{,}\AttributeTok{NHS=} \StringTok{\textquotesingle{}area\_name\textquotesingle{}}\NormalTok{) }\SpecialCharTok{\%\textgreater{}\%} 
  \FunctionTok{filter}\NormalTok{(NHS }\SpecialCharTok{!=}\StringTok{\textquotesingle{}Scotland\textquotesingle{}}\NormalTok{) }\SpecialCharTok{\%\textgreater{}\%} 
  \FunctionTok{mutate}\NormalTok{(}\AttributeTok{NHS =} \FunctionTok{sub}\NormalTok{(}\StringTok{"\^{}NHS "}\NormalTok{,}\StringTok{""}\NormalTok{, NHS))}
\FunctionTok{head}\NormalTok{(plot\_data)}
\end{Highlighting}
\end{Shaded}

\begin{verbatim}
## # A tibble: 6 x 3
##   NHS                  year number
##   <chr>               <dbl>  <dbl>
## 1 Ayrshire & Arran     2010  60822
## 2 Borders              2010  17226
## 3 Dumfries & Galloway  2010  22280
## 4 Fife                 2010  55334
## 5 Forth Valley         2010  43976
## 6 Grampian             2010  70337
\end{verbatim}

\#Analyse 14 Health Boards are too many to plot in the same
visualisation; the audience would be overwhelmed. So, I will create a
new column, classifying neighbouring NHS boards into Central Belt,
Borders, Highlands and Islands and North East. Then, I have to sum the
old NHS Board percentages for each year into a single value for the
Region for that year.

\begin{Shaded}
\begin{Highlighting}[]
\CommentTok{\#I wanted to use functions from the course but when I looked up how to recategorise a categorical variable, all the answers were using functions we hadn\textquotesingle{}t used. So, I asked chatgpt. First, it advised a statement for each NHS Board but I knew it could be done in groups and using | or \%in\% so I modified the prompt. }
\NormalTok{plot\_data }\OtherTok{\textless{}{-}}\NormalTok{ plot\_data }\SpecialCharTok{\%\textgreater{}\%} 
  \FunctionTok{mutate}\NormalTok{(}\AttributeTok{Region =} \FunctionTok{case\_when}\NormalTok{(}
\NormalTok{    NHS }\SpecialCharTok{\%in\%} \FunctionTok{c}\NormalTok{(}\StringTok{"Ayrshire \& Arran"}\NormalTok{ , }\StringTok{"Borders"}\NormalTok{ , }\StringTok{"Dumfries \& Galloway"}\NormalTok{) }\SpecialCharTok{\textasciitilde{}} \StringTok{"Borders"}\NormalTok{,}
\NormalTok{    NHS }\SpecialCharTok{\%in\%} \FunctionTok{c}\NormalTok{(}\StringTok{"Fife"}\NormalTok{ , }\StringTok{"Forth Valley"}\NormalTok{ , }\StringTok{"Greater Glasgow \& Clyde"}\NormalTok{ , }\StringTok{"Lanarkshire"}\NormalTok{ , }\StringTok{"Lothian"}\NormalTok{) }\SpecialCharTok{\textasciitilde{}} \StringTok{"Central Belt"}\NormalTok{,}
\NormalTok{    NHS }\SpecialCharTok{\%in\%} \FunctionTok{c}\NormalTok{(}\StringTok{"Grampian"}\NormalTok{ , }\StringTok{"Tayside"}\NormalTok{) }\SpecialCharTok{\textasciitilde{}} \StringTok{"North East"}\NormalTok{,}
\NormalTok{    NHS }\SpecialCharTok{\%in\%} \FunctionTok{c}\NormalTok{(}\StringTok{"Highland"}\NormalTok{ , }\StringTok{"Western Isles"}\NormalTok{ , }\StringTok{"Orkney"}\NormalTok{ , }\StringTok{"Shetland"}\NormalTok{) }\SpecialCharTok{\textasciitilde{}} \StringTok{"Highlands \& Islands"}\NormalTok{))}

\CommentTok{\#Now to calculate the regional percentages}
\NormalTok{summed\_data }\OtherTok{\textless{}{-}}\NormalTok{ plot\_data }\SpecialCharTok{\%\textgreater{}\%} 
  \FunctionTok{group\_by}\NormalTok{(Region, year) }\SpecialCharTok{\%\textgreater{}\%} 
  \FunctionTok{summarise}\NormalTok{(}\AttributeTok{total\_number =} \FunctionTok{sum}\NormalTok{(number))}
\end{Highlighting}
\end{Shaded}

\begin{verbatim}
## `summarise()` has grouped output by 'Region'. You can override using the
## `.groups` argument.
\end{verbatim}

\begin{Shaded}
\begin{Highlighting}[]
\NormalTok{summed\_data }\SpecialCharTok{\%\textgreater{}\%} 
  \FunctionTok{ggplot}\NormalTok{() }\SpecialCharTok{+}
  
  \FunctionTok{geom\_area}\NormalTok{(}\FunctionTok{aes}\NormalTok{(}\AttributeTok{x =}\NormalTok{ year, }\AttributeTok{y =}\NormalTok{ total\_number, }\AttributeTok{colour =}\NormalTok{ Region))}
\end{Highlighting}
\end{Shaded}

\includegraphics{prescriptions-for-a-d-p_files/figure-latex/Step 4: Communication-1.pdf}

\end{document}
